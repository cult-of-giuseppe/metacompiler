The goal of this document was to make you acquainted with the concept of inference systems. You have learned various aspects of logic and inference system, which we used to give rise to definitions of concepts such as numbers, boolean expressions, etc. Thanks to this we have seen how logical inference is actually self-sufficient to define all sorts of well-known mathematical concepts, without having to either:

\begin{inparaenum}[\itshape i\upshape)]
\item introduce these complex as primitive concepts, therefore making our formalism ``heavier'';
\item give them up and try to not use them.
\end{inparaenum}

Defining mathematical concepts could have gone further into most existing known aspects of mathematics, but instead after a while we have taken another route: defining whole programming languages as instances of logical inference. As a stepping stone we have defined a series of useful helpers, which are data structures such as lists or binary search trees. Then we have gone further to the definition of a small, yet whole, programming language.

This long trip has shown us not only the power of logic as a foundational tool for both mathematics and programming, but it has also shown the great potential of our logic-based language in the definition of programming languages.
